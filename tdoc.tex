\documentclass[11pt]{report}
\usepackage{tdoc}
\usepackage{float}
\floatstyle{boxed}
\restylefloat{figure}
\title{A git Tutorial}
\author{Matt Carvalho, Matt Gambogi, Rob Glossop}

\newcommand{\code}[1]{\texttt{#1}}

\begin{document}

% disable page numbering for titlepage
\thispagestyle{empty}
\maketitle

% restart page numbering in roman numerals for toc/figs
\clearpage \pagenumbering{roman} 

\tableofcontents

\listoffigures

% restart page numbering again for main body
\clearpage \pagenumbering{arabic}
\section{Intro}
This tutorial will introduce git. Some of the typical uses of git are as a version control system and as a distributed repository for a team of people all working on the same project. Version control software keeps track of all the changes that are made to files so you can view previous versions and roll back to them if necessary.  The distributed repository allows team members to all have local versions of the project files while keeping them in sync with the rest of the team. We will introduce the basics of using git in a multi-person project: how to track your work and sync it with other team members. We'll start with
the basics of how to create a repository and commit your own work into
it, then move on to merging, branches and other issues involved with a
multi-person git project. This document is intended for computer science students who have little or no experience using git. We assume you know the basics of the command line, and have access to the Computer Science machines or another machine with git installed.	

\section{A Note on Notation}
Code snippets and commands will be formatted in a \texttt{monospace} font or in
a separate figure.

Here is an example of a separate figure:
\begin{figure}[h]
    \caption{Example Command Snippet}
    \begin{lstlisting}
    $ example_command
    \end{lstlisting}
\end{figure}

\section{Procedures}
from the handout: "A procedures section must follow the Intro. The procedures section will consist of a series of sub-procedure sections, labeled with a short description of the sub-procedure to follow." -- I'm not sure what this is talking about...

\chapter{Single User \texttt{git}}
Git is well known as a distributed version control system but it is
firstly an excellent single user version control system.

Git tracks changes to collections of files known together as a repository.
When a change is made to the files in the repository, and a user would like
to save those changes, they "commit" them for later.


%multi-user tutorial
\chapter{Multi-user}
In this section you will be doing a demo of how to get code from
someone else's project and make changes to it on git. If you are
working on a project with a group or you want to make a contribution
to an open source project you may find some of these commands useful.
First you would cd to the directory that you want to put the project
folder into.  For this demo you can just clone it to your home
directory.  Now type

\begin{figure}[H]
  \begin{lstlisting}
    $ git clone [url of project]
  \end{lstlisting}
\end{figure}


to download all the files of that project to your computer.  If you
make a change and push it now you will be making changes to the master
branch Create a new branch by typing

\begin{figure}[H]
  \begin{lstlisting}
    $ git branch [new_branch_name]
  \end{lstlisting}
\end{figure}

You can verify that the branch was created with

\begin{figure}[H]
  \begin{lstlisting}
    $ git branch
  \end{lstlisting}
\end{figure}

And switch to this branch with

\begin{figure}[H]
  \begin{lstlisting}
    $ git checkout [new_branch_name]
  \end{lstlisting}
\end{figure}

Make a change to a file, add it, and commit it.

\begin{figure}[H]
  \begin{lstlisting}
    $ git add [filename]
    $ git commit -m "commit message"
    \end{lstlisting}
\end{figure}

You just committed the changes to your local branch.  You can push
this branch to the repository with

\begin{figure}[H]
  \begin{lstlisting}
    $ git push -u origin [new_branch_name]
  \end{lstlisting}
\end{figure}

\begin{thebibliography}{9}

\bibitem{gitscm}
    Git. (n.d.). Retrieved \today, from http://www.git-scm.com/

\end{thebibliography}
\end{document}
