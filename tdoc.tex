\documentclass[11pt]{report}
\usepackage{tdoc}
\usepackage{float}
\floatstyle{boxed}
\restylefloat{figure}
\title{A git Tutorial}
\author{Matt Carvalho, Matt Gambogi, Rob Glossop}

\newcommand{\code}[1]{\texttt{#1}}

\begin{document}

% disable page numbering for titlepage
\thispagestyle{empty}
\maketitle

% restart page numbering in roman numerals for toc/figs
\clearpage \pagenumbering{roman} 

\tableofcontents

\listoffigures

% restart page numbering again for main body
\clearpage \pagenumbering{arabic}
\section{Intro}
This tutorial will introduce git.  Git is a version control system,
used both for individual work and projects distributed among multiple
people.  Version control software keeps track of all the changes that
are made to files so you can view previous versions and roll back to
them if necessary.  A distributed version control system allows team
members to all have local versions of the project files while keeping
them in sync with the rest of the team.

We will introduce the basics of using git in a multi-person project:
how to track your work and sync it with other team members. We'll
start with the basics of how to create a repository and commit your
own work into it, then move on to merging, branches and other issues
involved with a multi-person git project. This document is intended
for computer science students who have little or no experience using
git. We assume you know the basics of the command line, and have
access to the Computer Science machines or another machine with git
installed.

\section{A Note on Notation}
Code snippets and commands will be formatted in a \texttt{monospace} font or in
a separate figure. Placeholders are indicated with \texttt{[square
    braces]}. Commands will have a \texttt{\$} prefix, while output
will have nothing.
\begin{figure}[H]
  \caption{Example Command Snippet}
  \begin{lstlisting}
    > example_command
    Success!
  \end{lstlisting}
\end{figure}

\chapter{Single User \texttt{git}}
Git is well known as a distributed version control system but it is
firstly an excellent single user version control system.

Git tracks changes to collections of files known together as a repository.
When a change is made to the files in the repository, and a user would like
to save those changes, they "commit" them for later. Since you are a CS student,
you have access to the CS machines, so you should be able to sign in to any and
try these examples.

\begin{figure}[H]
  \caption{ssh into the CS Machines}
  \begin{lstlisting}
    > ssh abc1234@glados.cs.rit.edu
    Password:
  \end{lstlisting}
\end{figure}

Once you've logged in, verify that git is installed:

\begin{figure}[H]
  \lstinputlisting{git_help.txt}
\end{figure}

\begin{figure}[H]
  \caption{verify}
  \lstinputlisting{git_help_2.txt}
\end{figure}

\section{Committing}

Git is used in the context of a \emph{repository}, a collection of
files you want to track. To create a repository, go to the directory
with your files and run

\begin{figure}[H]
  \begin{lstlisting}
    > git init
    Initialized empty git repository in
    /home/robert/profcomm/team-document/.git
  \end{lstlisting}
\end{figure}

The repository is ready to use, but it doesn't have any files in it
yet. To add files, run

\begin{figure}[H]
  \begin{lstlisting}
    > git add homework1.c
  \end{lstlisting}
\end{figure}

TODO GIT STATUS MENTION

Your file isn't tracked yet: it's in the \emph{index}, git's staging
area. From here, \texttt{git commit} will package all the files in the
index into a commit.

\begin{figure}[H]
  \begin{lstlisting}
    > git commit -m "Fixed diagonal line drawing"
    [master cb8a72a] Fixed diagonal line drawing
     1 file changed, 95 insertions(+), 22 deletions(-)
  \end{lstlisting}
\end{figure}

\section{History}

Git commits are organized into a history: each commit has a parent,
leading back to the beginning of the repository. You can view the
history with \texttt{git log}:

\begin{figure}[H]
  \begin{lstlisting}
    > git log
    commit ed7c16f565aa0cdd09ca02f6fd653d8704b4d93b
    Author: Robert Glossop <robgssp@gmail.com>
    Date:   Wed Oct 7 22:40:40 2015 -0400
    
        split up help
    
    commit 9c8d30d2c5b73b63aa6d3401519218d5b38b8379
    Author: Matt Gambogi <m@gambogi.com>
    Date:   Wed Oct 7 22:31:27 2015 -0400
    
        added a few things
    
    commit 7351c1145cb10280101ef13e6225914b5789dd7f
    Author: Matt Gambogi <m@gambogi.com>
    Date:   Wed Oct 7 22:09:08 2015 -0400
    
        took a book off a hook out of math mode

    ...
  \end{lstlisting}
\end{figure}

Notice the long string above each commit: this is a \emph{commit
  hash}, which is generated from the commit's contents and used by git
command to uniquely identify commits.

%multi-user tutorial
\chapter{Multi-user}
In this section you will be doing a demo of how to get code from
someone else's project and make changes to it on git. If you are
working on a project with a group or you want to make a contribution
to an open source project you may find some of these commands useful.
First you would cd to the directory that you want to put the project
folder into.  For this demo you can just clone it to your home
directory.  Now type
\begin{figure}[H]
  \caption{Example git clone}
  \begin{lstlisting}
    > git clone [url of project]
  \end{lstlisting}
\end{figure}

to download all the files of that project to your computer.  If you
make a change and push it now you will be making changes to the master
branch Create a new branch by typing

\begin{figure}[H]
  \caption{Example git branch}
  \begin{lstlisting}
    > git branch ANewChange
  \end{lstlisting}
\end{figure}

You can verify that the branch was created with

\begin{figure}[H]
  \caption{Example git branch listing}
  \begin{lstlisting}
    > git branch
      ANewChange
    * master
  \end{lstlisting}
\end{figure}

And switch to this branch with

\begin{figure}[H]
  \caption{Example git checkout}
  \begin{lstlisting}
    > git checkout ANewChange
  \end{lstlisting}
\end{figure}

Make a change to a file, add it, and commit it.

\begin{figure}[H]
  \caption{Example git commit}
  \begin{lstlisting}
    > git add codezone.c
    > git commit -m "Code Zone expanded"
    \end{lstlisting}
\end{figure}

You just committed the changes to your local branch.  You can push
this branch to the repository with

\begin{figure}[H]
  \caption{Example git push}
  \begin{lstlisting}
    > git push -u origin ANewChange
  \end{lstlisting}
\end{figure}

The -u option allows you to push to others' currently checked-out
branches. Use it with care.

Similarly, you can pull other people's changes with

\begin{figure}[H]
  \begin{lstlisting}
    > git pull origin ANewChange
  \end{lstlisting}
\end{figure}

\section{Merging}

Once you've worked on a branch for awhile, you'll want to merge it
back to the main (master) branch. Switch back to master then run
\texttt{git merge}:

\begin{figure}[H]
  \begin{lstlisting}
    > git checkout master
    Switched to branch 'master'
    > git merge ANewChange
  \end{lstlisting}
\end{figure}

From here, a few things can happen. If there haven't been any changes
on master, the merge will simply fast-forward and complete, and the
history won't show a branch at all.

\begin{figure}[H]
  \begin{lstlisting}
    git merge ANewChange
    Updating ed7c16f..b80a9ee
    Fast-forward
     homework1.c | 163 +++++++++++++++++++++++++++++++++++------
     1 file changed, 141 insertions(+), 22 deletions(-)
  \end{lstlisting}
\end{figure}

If there were changes in master but they don't overlap, you won't have
to resolve merge conflicts but git will prompt you to create a merge
commit:

TODO EXAMPLE HERE

This commit will be created with two parents, one for each side of the
branch. You can view a branching history with

\begin{figure}[H]
  \begin{lstlisting}
    > git log --graph --oneline --decorate
    *   363b2f3 Merge branch 'master' of
    |\          github.com:gambogi/TeamDocument
    | * A0b7a28 forgot to kill another line
    | * d0b0039 not a draft :|
    * | b1129f1 line wrap
    |/  
    * a262446 updated pdf
    * b42f2de added a word on notation
    ...
  \end{lstlisting}
\end{figure}

If the changes in master and your branch overlap, git will prompt you
to resolve the errors. The offending files will be marked with

\begin{figure}[H]
  \begin{lstlisting}
    >>>>>>>>>>
    one file's changes
    ==========
    other file's changes
    <<<<<<<<<<
  \end{lstlisting}
\end{figure}

\begin{thebibliography}{9}

\bibitem{gitscm}
    Git. (n.d.). Retrieved \today, from http://www.git-scm.com/

\end{thebibliography}
\end{document}


