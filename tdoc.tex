\documentclass[11pt]{report}
\usepackage{tdoc}
\usepackage{float}
\floatstyle{boxed}
\restylefloat{figure}
\title{A git Tutorial}
\author{Matt Carvalho, Matt Gambogi, Rob Glossop}

\newcommand{\code}[1]{\texttt{#1}}

\begin{document}

% disable page numbering for titlepage
\thispagestyle{empty}
\maketitle

% restart page numbering in roman numerals for toc/figs
\clearpage \pagenumbering{roman} 

\tableofcontents

\listoffigures

% restart page numbering again for main body
\clearpage \pagenumbering{arabic}
\section{Intro}
This tutorial will introduce git.  Git is a version control system,
used both for individual work and projects distributed among multiple
people.  Version control software keeps track of all the changes that
are made to files so you can view previous versions and roll back to
them if necessary.  A distributed version control system allows team
members to all have local versions of the project files while keeping
them in sync with the rest of the team.

We will introduce the basics of using git in a multi-person project:
how to track your work and sync it with other team members. We'll
start with the basics of how to create a repository and commit your
own work into it, then move on to merging, branches and other issues
involved with a multi-person git project. This document is intended
for computer science students who have little or no experience using
git. We assume you know the basics of the command line, and have
access to the Computer Science machines or another machine with git
installed.

\section{A Note on Notation}
Code snippets and commands will be formatted in a \texttt{monospace} font or in
a separate figure. Placeholders are indicated with \texttt{[square
    braces]}. Commands will have a \texttt{>} prefix, while output
will have nothing.
\begin{figure}[H]
  \caption{Example Command Snippet}
  \begin{lstlisting}
    > example_command
    Success!
  \end{lstlisting}
\end{figure}

\chapter{Single User \texttt{git}}
Git is well known as a distributed version control system but it is
firstly an excellent single user version control system. 
Git tracks changes to collections of files known together as a repository.
When a change is made to the files in the repository, and a user would like
to save those changes, they "commit" them for later.

\section{Trying git}
Since you are a CS student,
you have access to the CS machines, so you should be able to sign in to any and
try these examples.

\begin{figure}[H]
  \caption{ssh into the CS Machines}
  \begin{lstlisting}
    > ssh abc1234@glados.cs.rit.edu
    Password:
  \end{lstlisting}
\end{figure}

Once you've logged in, verify that git is installed:

\begin{figure}[H]\lstinputlisting{git_help.txt}\end{figure}
\begin{figure}[H]\caption{Verify installation}\lstinputlisting{git_help_2.txt}\end{figure}

Let's create a repo. First, create a directory:
\begin{figure}[H]
  \begin{lstlisting}
    > mkdir example_git
  \end{lstlisting}
\end{figure}

Great! Let's initalize a git repository in this folder:
\begin{figure}[H]
  \caption{Initializing a Repo}
  \begin{lstlisting}
    > cd example_git
    > git init
    Initialized empty Git repository in /path/to/example_git/.git
  \end{lstlisting}
\end{figure}

Now let's create a file and commit it:
\begin{figure}[H]
  \caption{Making your first commit}
  \begin{lstlisting}
    > touch README
    > git add README
    > git commit -m "Initial commit"
    [master (root-commit) fffffff] Initial commit
     1 file changed, 0 insertions(+), 0 deletions(-)
     create mode 100644 README
  \end{lstlisting}
\end{figure}

Let's break down these commands. First we \texttt{git add} a file,
which puts it in the \emph{index}. \texttt{git commit} packages the
files in the index into a commit, which is given a message.

The output of \texttt{git commit} also bears looking
at. \texttt{master} refers to the name of the branch we just committed
to. \texttt{(root-commit)} indicates this was the first commit in the
repository. \texttt{ffffff} (or similarly positioned hex value) is the
hash of the ppfiles in the repository. \texttt{Initial commit} is the
title of your commit message. These messages should be short, and
capture the essence of what changed between this commit and the
last. \texttt{1 file changed, 0 insertions(+), 0 deletions(-)} is just
metadata describing what happened. Similarly, the last line just
denotes that \texttt{README} is now being tracked by git.

This output probably raises the question: what is a branch?  If a
series of commits forms a linear picture of the history of changes,
branches are divergent histories of change. Git easily allows for the
creation of branches and provides tools for unifying them.

To see a list of branches try typing 
\begin{figure}[H]
  \caption{Example branch listing}
  \begin{lstlisting}
  > git branch
  * master
  \end{lstlisting}
 \end{figure}

Let's create a branch:
\begin{figure}[H]
  \caption{Example branch creation}
  \begin{lstlisting}
  > git checkout -b twig
  Switched to a new branch 'twig'
  \end{lstlisting}
\end{figure}

Now if we run \texttt{git branch} again, we should see a list of
branches. This should be just \texttt{master} and \texttt{twig}.
The name \texttt{twig} is just an example, branches may be called
whatever the user desires. A \texttt{checkout} switches your working
directory to that of the target branch. Try editing the \texttt{README}
and committing it. Then switch branches and check again.

Now let's say you want to get the changes made in a branch into some other
branch. You'll need to perform what is known in the git world as a
merge.

\section{Merging}

Once you've worked on a branch for awhile, you'll want to merge it
back to the master branch. Switch back to master then run
\texttt{git merge}:

\begin{figure}[H]
  \begin{lstlisting}
    > git checkout master
    Switched to branch 'master'
    > git merge twig
  \end{lstlisting}
\end{figure}

From here, a few things can happen. If there haven't been any changes
on master, the merge will simply fast-forward and complete, and the
history won't show a branch at all.

\begin{figure}[H]
  \begin{lstlisting}
    > git merge twig
    Updating ed7c16f..b80a9ee
    Fast-forward
     README | 163 +++++++++++++++++++++++++++++++++++------
     1 file changed, 141 insertions(+), 22 deletions(-)
  \end{lstlisting}
\end{figure}

If there were changes in master but they don't overlap, you won't have
to resolve merge conflicts but git will prompt you to create a merge
commit:

\begin{figure}[H]
  \begin{lstlisting}
    Merge branch 'twig' into master
    
    # It looks like you may be committing a merge.
    # If this is not correct, please remove the file
    #  .git/MERGE_HEAD
    # and try again.
  \end{lstlisting}
\end{figure}

This commit will be created with two parents, one for each side of the
branch. You can view a branching history with

\begin{figure}[H]
  \begin{lstlisting}
    > git log --graph --oneline --decorate
    *   363b2f3 Merge branch 'master' of
    |\          github.com:gambogi/TeamDocument
    | * A0b7a28 forgot to kill another line
    | * d0b0039 not a draft :|
    * | b1129f1 line wrap
    |/  
    * a262446 updated pdf
    * b42f2de added a word on notation
    ...
  \end{lstlisting}
\end{figure}

If the changes in master and your branch overlap, git will prompt you
to resolve the errors. The offending files will be marked with

\begin{figure}[H]
  \begin{lstlisting}
    <<<<<<<<<<
    one file's changes
    ==========
    other file's changes
    >>>>>>>>>>
  \end{lstlisting}
\end{figure}

Resolve the issues then add and commit the file, and you're done.

\section{History}

Git commits are organized into a history: each commit has a parent,
leading back to the beginning of the repository. You can view the
history with \texttt{git log}:

\begin{figure}[H]
  \begin{lstlisting}
    > git log
    commit ed7c16f565aa0cdd09ca02f6fd653d8704b4d93b
    Author: Robert Glossop <robgssp@gmail.com>
    Date:   Wed Oct 7 22:40:40 2015 -0400
    
        split up help
    
    commit 9c8d30d2c5b73b63aa6d3401519218d5b38b8379
    Author: Matt Gambogi <m@gambogi.com>
    Date:   Wed Oct 7 22:31:27 2015 -0400
    
        added a few things
    
    commit 7351c1145cb10280101ef13e6225914b5789dd7f
    Author: Matt Gambogi <m@gambogi.com>
    Date:   Wed Oct 7 22:09:08 2015 -0400
    
        took a book off a hook out of math mode

    ...
  \end{lstlisting}
\end{figure}

Notice the long string above each commit: this is a \emph{commit
  hash}, which is generated from the commit's contents and used by git
command to uniquely identify commits.

It's sometimes useful to view what changed between commits.
\texttt{git diff} will show diffs between file revisions, be it your
working tree, other branches or previous commit. Running \texttt{git
  diff} with no arguments will show your unstaged changes. \texttt{git
  diff 9c8d30} will show the changes between commit \texttt{9c8d30}
and now. \texttt{git diff 9c8d30..abc123} will show all the changes
from the first commit to the second. Finally, \texttt{git diff
  abc123...def456} will show the diff from the common ancestor of the
two to the second. This is useful to show what's changed in your
branch when the base branch has also advanced, e.g. \texttt{git diff
  twig...master}.

% multi-user tutorial
\chapter{Multi-user \texttt{git}}

In this section you will be doing a demo of how to get code from
someone else's project and make changes to it on git. If you are
working on a project with a group or you want to make a contribution
to an open source project you may find some of these commands useful.
First you would cd to the directory that you want to put the project
folder into.  For this demo you can just clone it to your home
directory.

There's an example repository hosted on the CS machines. To clone it:

\begin{figure}[H]
  \caption{Example git clone}
  \begin{lstlisting}
      # on your machine
    > git clone abc1234@glados.cs.rit.edu:~rig1635/test-repo
    remote: Counting objects: 96, done.
    remote: Total 96 (delta 42), reused 0 (delta 0), done.
    Unpacking objects: 100% (96/96), done.
    Checking connectivity... done.

      # on a CS machine
    > git clone ~rig1635/test-repo
  \end{lstlisting}
\end{figure}

From here you can work on your local repository; commands like
\texttt{commit} and \texttt{branch} don't touch the remote (referred
to as ``origin''.

\begin{figure}[H]
  \caption{Example git branch}
  \begin{lstlisting}
    > git branch ANewChange
  \end{lstlisting}
\end{figure}

You can verify that the branch was created with

\begin{figure}[H]
  \caption{Example git branch listing}
  \begin{lstlisting}
    > git branch
      ANewChange
    * master
  \end{lstlisting}
\end{figure}

And switch to this branch with

\begin{figure}[H]
  \caption{Example git checkout}
  \begin{lstlisting}
    > git checkout ANewChange
  \end{lstlisting}
\end{figure}

Make a change to a file, add it, and commit it.

\begin{figure}[H]
  \caption{Example git commit}
  \begin{lstlisting}
    > git add codezone.c
    > git commit -m "Code Zone expanded"
    \end{lstlisting}
\end{figure}

You just committed the changes to your local branch.  You can push
this branch to the repository with

\begin{figure}[H]
  \caption{Example git push}
  \begin{lstlisting}
    > git push -u origin ANewChange
  \end{lstlisting}
\end{figure}

The -u option allows you to push to others' currently checked-out
branches. Use it with care.

Similarly, you can pull other people's changes with

\begin{figure}[H]
  \begin{lstlisting}
    > git pull origin ANewChange
  \end{lstlisting}
\end{figure}

\begin{thebibliography}{9}

\bibitem{gitscm}
    Git. (n.d.). Retrieved \today, from http://www.git-scm.com/

\end{thebibliography}
\end{document}
