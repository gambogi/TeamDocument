\documentclass[11pt]{report}
\usepackage{tdoc}
\usepackage{float}
\floatstyle{boxed}
\restylefloat{figure}
\title{A git Tutorial}
\author{Matt Carvalho, Matt Gambogi, Rob Glossop}

\newcommand{\code}[1]{\texttt{#1}}

\begin{document}

\section*{Usability Study}
Matt Carvalho, Matt Gambogi, Rob Glossop


Test Plan:

We are going to test our document by giving it to freshmen computer
science students who have no experience with git. We will have them
ssh into their CS accounts so we can be sure they all have access to
git.
        
Some of the features in our document that our subjects should be able
to successfully execute are:

\begin{itemize}
\item Initializing a git repository.

\item Making the initial commit to the repo.

\item Creating a new branch and merging that branch with master (both when
  the branch has merging problems and when it doesn’t).
  
\item Cloning our example repository and making changes to it.
\end{itemize}
In order to gauge the effectiveness of our instructions we will make
sure our subjects can complete these tasks without any errors. If they
end up having problems we will reword the specific example or add more
instructions for what to do if something goes wrong.


Feedback
\begin{itemize}
\item initial commit message looked different -- give instructions for git
config

\item merge twig to branch - make it more clear to make changes to README
and add/commit

\item they only were reading the text in the code boxes -- make it so
  everything necessary to complete tutorial is in code boxes
  
\item add git tree to the introduction

\item abc1234?

\item what’s a tag?

\item multi-user -- make it clear to cd to a different directory

\item explain how to delete branches
\end{itemize}

\end{document}
