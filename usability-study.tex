\documentclass[11pt]{report}
\usepackage{tdoc}
\usepackage{float}
\floatstyle{boxed}
\restylefloat{figure}
\title{A git Tutorial}
\author{Matt Carvalho, Matt Gambogi, Rob Glossop}

\newcommand{\code}[1]{\texttt{#1}}

\begin{document}

\section*{Usability Study}
Matt Carvalho, Matt Gambogi, Rob Glossop


Test Plan:

We are going to test our document by giving it to freshmen computer
science students who have no experience with git. We will have them
ssh into their CS accounts so we can be sure they all have access to
git.

Some of the features in our document that our subjects should be able
to successfully execute are:

\begin{itemize}
\item Initializing a git repository.

\item Making the initial commit to the repo.

\item Creating a new branch and merging that branch with master (both when
  the branch has merging problems and when it doesn’t).
  
\item Cloning our example repository and making changes to it.
\end{itemize}
In order to gauge the effectiveness of our instructions we will make
sure our subjects can complete these tasks without any errors. If they
end up having problems or questions about a specific instruction
we will reword the example or add more instructions for what to do if something goes wrong.\\


\section{Insights from Feedback}
One thing we didn't think of is that if the person hasn't used git before they won't
have certain things like a username and email set, so we added an example of git config for
username and email. One of our subjects wasn't getting the same output as the example 
for merging two branches. This was because they hadn't made any changes in the new branch,
so we made it more clear that they need to change a file and commit it to the new branch 
before merging. In our multi-user section when a user was trying to clone a repository that we 
set up in one of our CS accounts they were trying to sign in with the example account that we gave
(abc1234) so we made it clear to replace this with their actual username. Also in the multi-user section - 
one subject cloned the example repo inside the directory they were using for the first example, creating
a nested git repo, so we added the instruction to cd back to their home directory first. Someone
wanted to know how they could delete a branch after creating it, so we added an example for that.

\end{document}
